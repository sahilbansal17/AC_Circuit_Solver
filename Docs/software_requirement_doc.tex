\documentclass[]{article}
\usepackage[left=0.7in,right=0.7in,top=0.9in,bottom=0.4in]{geometry}

%opening
\title{\textbf{COP290 : Assignment 2 - AC Circuit Solver\\Software Requirement Specification}}
\author{Sahil Bansal(2016CSJ0008) \& Sahil Singh(2016CSJ0025)}
	
\begin{document}

\maketitle
\vspace{1ex}
\paragraph{
	\LARGE{Functionalities involved in the software:}\newline
}
\Large
\begin{itemize}
	\item The circuit is drawn in \textbf{SVG} format by drawing individual nets as vertical lines and then connecting the components appropriately. Overlapping is avoided and a properly designed algorithm is implemented.
	\item The components are clickable through the text showing the name of the component.
	\item Most of the syntax and semantic errors are handled and shown on the terminal. This was possible due to the usage of lex and yacc to parse the netlist. 
	\item The solving part is handled using the \textbf{Eigen} library in C++ along with \textbf{complex} header file. \textbf{Modified Nodal Analysis} algorithm is used for solving. 
	\item After solving, the output is displayed in a text file and is in the increasing order of frequencies of the sources. The magnitude and phase values are rounded upto 3 decimal places.
\end{itemize}

\end{document}
